\documentclass[10pt, letterpaper]{article}

% Packages:
\usepackage[
    ignoreheadfoot, % set margins without considering header and footer
    top=2 cm, % seperation between body and page edge from the top
    bottom=2 cm, % seperation between body and page edge from the bottom
    left=2 cm, % seperation between body and page edge from the left
    right=2 cm, % seperation between body and page edge from the right
    footskip=1.0 cm, % seperation between body and footer
    % showframe % for debugging 
]{geometry} % for adjusting page geometry
\usepackage{titlesec} % for customizing section titles
\usepackage{tabularx} % for making tables with fixed width columns
\usepackage{array} % tabularx requires this
\usepackage[dvipsnames]{xcolor} % for coloring text
\definecolor{primaryColor}{RGB}{0, 0, 0} % define primary color
\usepackage{enumitem} % for customizing lists
\usepackage{fontawesome5} % for using icons
\usepackage{amsmath} % for math
\usepackage[
    pdftitle={CV de Williams Medina},
    pdfauthor={Williams Medina},
    colorlinks=true,
    urlcolor=primaryColor
]{hyperref} % for links, metadata and bookmarks
\usepackage[pscoord]{eso-pic} % for floating text on the page
\usepackage{calc} % for calculating lengths
\usepackage{bookmark} % for bookmarks
\usepackage{lastpage} % for getting the total number of pages
\usepackage{changepage} % for one column entries (adjustwidth environment)
\usepackage{paracol} % for two and three column entries
\usepackage{ifthen} % for conditional statements
\usepackage{needspace} % for avoiding page brake right after the section title
\usepackage{iftex} % check if engine is pdflatex, xetex or luatex

% Ensure that generate pdf is machine readable/ATS parsable:
\ifPDFTeX
    \input{glyphtounicode}
    \pdfgentounicode=1
    \usepackage[T1]{fontenc}
    \usepackage[utf8]{inputenc}
    \usepackage{lmodern}
\fi

\usepackage{charter}

% Some settings:
\raggedright
\AtBeginEnvironment{adjustwidth}{\partopsep0pt} % remove space before adjustwidth environment
\pagestyle{empty} % no header or footer
\setcounter{secnumdepth}{0} % no section numbering
\setlength{\parindent}{0pt} % no indentation
\setlength{\topskip}{0pt} % no top skip
\setlength{\columnsep}{0.15cm} % set column seperation
\pagenumbering{gobble} % no page numbering

\titleformat{\section}{\needspace{4\baselineskip}\bfseries\large}{}{0pt}{}[\vspace{1pt}\titlerule]

\titlespacing{\section}{
    % left space:
    -1pt
}{
    % top space:
    0.3 cm
}{
    % bottom space:
    0.2 cm
} % section title spacing

\renewcommand\labelitemi{$\vcenter{\hbox{\small$\bullet$}}$} % custom bullet points
\newenvironment{highlights}{
    \begin{itemize}[
        topsep=0.10 cm,
        parsep=0.10 cm,
        partopsep=0pt,
        itemsep=0pt,
        leftmargin=0 cm + 10pt
    ]
}{
    \end{itemize}
} % new environment for highlights


\newenvironment{highlightsforbulletentries}{
    \begin{itemize}[
        topsep=0.10 cm,
        parsep=0.10 cm,
        partopsep=0pt,
        itemsep=0pt,
        leftmargin=10pt
    ]
}{
    \end{itemize}
} % new environment for highlights for bullet entries

\newenvironment{onecolentry}{
    \begin{adjustwidth}{
        0 cm + 0.00001 cm
    }{
        0 cm + 0.00001 cm
    }
}{
    \end{adjustwidth}
} % new environment for one column entries

\newenvironment{twocolentry}[2][]{
    \onecolentry
    \def\secondColumn{#2}
    \setcolumnwidth{\fill, 4.5 cm}
    \begin{paracol}{2}
}{
    \switchcolumn \raggedleft \secondColumn
    \end{paracol}
    \endonecolentry
} % new environment for two column entries

\newenvironment{threecolentry}[3][]{
    \onecolentry
    \def\thirdColumn{#3}
    \setcolumnwidth{, \fill, 4.5 cm}
    \begin{paracol}{3}
    {\raggedright #2} \switchcolumn
}{
    \switchcolumn \raggedleft \thirdColumn
    \end{paracol}
    \endonecolentry
} % new environment for three column entries

\newenvironment{header}{
    \setlength{\topsep}{0pt}\par\kern\topsep\centering\linespread{1.5}
}{
    \par\kern\topsep
} % new environment for the header

\newcommand{\placelastupdatedtext}{% \placetextbox{<horizontal pos>}{<vertical pos>}{<stuff>}
  \AddToShipoutPictureFG*{% Add <stuff> to current page foreground
    \put(
        \LenToUnit{\paperwidth-2 cm-0 cm+0.05cm},
        \LenToUnit{\paperheight-1.0 cm}
    ){\vtop{{\null}\makebox[0pt][c]{
        \small\color{gray}\textit{Last updated in September 2024}\hspace{\widthof{Last updated in September 2024}}
    }}}%
  }%
}%

% save the original href command in a new command:
\let\hrefWithoutArrow\href

% new command for external links:


\begin{document}
    \newcommand{\AND}{\unskip
        \cleaders\copy\ANDbox\hskip\wd\ANDbox
        \ignorespaces
    }
    \newsavebox\ANDbox
    \sbox\ANDbox{$|$}

    \begin{header}
        \fontsize{25 pt}{25 pt}\selectfont Williams Medina

        \vspace{5 pt}

        \normalsize
        \mbox{Córdoba, Argentina}%
        \kern 5.0 pt%
        \AND%
        \kern 5.0 pt%
        \mbox{\hrefWithoutArrow{mailto:wmedina685@gmail.com}{wmedina685@gmail.com}}%
        \kern 5.0 pt%
        \AND%
        \kern 5.0 pt%
        \mbox{\hrefWithoutArrow{tel:+90-541-999-99-99}{(+54) 123 456 7890}}%
        \kern 5.0 pt%
        \AND%
        \kern 5.0 pt%
    
        \mbox{\hrefWithoutArrow{https://linkedin.com/in/wamedina}{linkedin.com/in/wamedina}}%
        \kern 5.0 pt%
        \AND%
        \kern 5.0 pt%
        \mbox{\hrefWithoutArrow{https://github.com/williamsmedina}{github.com/williamsmedina}}%
    \end{header}

    \vspace{5 pt - 0.3 cm}


    \section{Resumen}



        
        \begin{onecolentry}
            Me dedico a la ciberseguridad, con más de 4 años de experiencia en IT. He trabajado con tecnologías como Microsoft Azure, Windows Server, Fortigate, Kaspersky Security Center y Palo Alto Cortex XDR, participando en proyectos de soporte y consultoría en entornos on-premises y en la nube. Poseo sólidos conocimientos en protocolos de red (DNS, SMB, SMTP, HTTP, TCP/IP) y actualmente me encuentro abocado a la securización de puestos de trabajo e infraestructuras Cloud y On-Premises, así como a su gobernanza. Me destaco por mi capacidad para resolver desafíos complejos de manera proactiva y colaborativa.
        \end{onecolentry}

        \vspace{0.2 cm}   

    \section{Educación}

\begin{twocolentry}{
    2021 – 2022
}
    \textbf{Colegio Universitario IES}, Analista de Sistemas
\end{twocolentry}

\vspace{0.10 cm}
\begin{onecolentry}
    \begin{highlights}
        % (Agrega detalles adicionales, logros o cursos relevantes aquí, si lo consideras necesario)
    \end{highlights}
\end{onecolentry}




    
   \section{Experiencia laboral}
\begin{twocolentry}{
    Dic. 2024 – Presente
}
    \textbf{Protection \& Response Specialist}, Vulps - Grupo CEDI
\end{twocolentry}

\vspace{0.10 cm}
\begin{onecolentry}
    \begin{highlights}
         \item Administro la consola antivirus de Palo Alto Cortex XDR, gestionando más de 13,000 endpoints (servidores Windows y Linux, estaciones de trabajo y dispositivos móviles) distribuidos en más de 100 sitios para proporcionar la protección de todo el entorno.
        \item Realizar análisis de malware en incidentes complejos y llevar a cabo acciones de remediación, minimizando el impacto de amenazas avanzadas.
        \item Participo en pruebas de concepto (PoC) y en el despliegue de soluciones de ciberseguridad, apoyando la integración de nuevas tecnologías en entornos críticos.
        \item Asesoro en la implementación de medidas de seguridad en Azure y Microsoft Entra ID, aplicando buenas prácticas de acceso condicional, principio de menor privilegio, y gobernanza de la nube.
        \item Doy seguimiento a proyectos de ciberseguridad, asegurando la correcta integración y funcionamiento de diversas soluciones tecnológicas.
        \item Brindo apoyo técnico especializado en la resolución de problemas complejos donde interactuaban múltiples sistemas, optimizando la respuesta ante incidentes.
    \end{highlights}
\end{onecolentry}

\vspace{0.2 cm}   

\begin{twocolentry}{
    Ene. 2024 – Nov. 2024
}
    \textbf{Support Engineer}, CEDI Tech Consulting
\end{twocolentry}

\vspace{0.10 cm}
\begin{onecolentry}
    \begin{highlights}
        \item Brindé soporte y consultoría a áreas de sistemas, satisfaciendo de manera efectiva las necesidades de clientes de todos los tamaños.
        \item Asesoré en el manejo de tecnologías de vanguardia—Microsoft Azure, Microsoft 365, Windows Server (AD DS, AD CS, File Server Services, Exchange, IIS, etc.), Kaspersky Security Center, VMware ESXi y vSphere, Fortigate y Dell EMC SmartFabric OS—acompañando a los equipos de TI en la identificación y resolución de incidencias para optimizar infraestructuras complejas.
    \end{highlights}
\end{onecolentry}

\vspace{0.2 cm}

\begin{twocolentry}{
    Dic. 2021 – Dic. 2023
}
    \textbf{Soporte IT}, Radio Mitre S.A - Grupo Clarín
\end{twocolentry}

\vspace{0.10 cm}
\begin{onecolentry}
    \begin{highlights}
        \item Automatización, gestión, documentación y auditoría de procesos críticos en Active Directory y M365, abarcando altas, bajas, modificaciones de usuarios, grupos de seguridad, MFA y OUs, lo que permitió optimizar la eficiencia operativa y fortalecer la gestión de identidades.
        \item Implementé herramientas para la integración de plataformas en productos ManageEngine con Microsoft Entra ID/Active Directory y OAuth, potenciando la interoperabilidad entre sistemas.
        \item Administré la seguridad de más de 400 estaciones de trabajo a través de Kaspersky Security Center, diseñando e implementando políticas de seguridad personalizadas que respondieran a las necesidades del negocio y automatizando las actualizaciones de sistemas y software de terceros para asegurar una protección integral y continuidad operativa
        \item Operé firewalls Fortigate para más de 2000 hosts distribuidos en tres sitios, gestionando address policies, NAT, IDS/IPS, SD-WAN Rules y VPN, e integré Lumu para una respuesta eficaz ante incidentes.
        \item Controlé y mantuve la operatividad de Exchange Server, DHCP, DNS, IIS y recursos compartidos (net shares y print servers) utilizando RDP, VNC, MMC y PowerShell, asegurando la continuidad de los servicios críticos.
        \item Monitoreé la infraestructura IT y radial con PRTG, configurando dashboards y webhooks (integrados con MS Teams) para identificar, escalar y prevenir fallas de forma proactiva.
    \end{highlights}
\end{onecolentry}

\vspace{0.2 cm}

\begin{twocolentry}{
    Ago. 2020 – Nov. 2021
}
    \textbf{Técnico de Soporte de TI}, Telenex
\end{twocolentry}

\vspace{0.10 cm}
\begin{onecolentry}
    \begin{highlights}
        \item Proporcioné soporte técnico de nivel 1, tanto presencial como remoto, a más de 200 usuarios finales en entornos Windows, resolviendo incidencias de forma ágil y eficaz.
        \item Administré y optimicé Active Directory, servidores de impresión, DHCP Mikrotik y Google Workspace, asegurando la continuidad y el rendimiento de la infraestructura IT.
        \item Implementé infraestructura física de IT, abarcando el cableado de red y el montaje en rack de equipos, lo que contribuyó a mejorar la conectividad y estabilidad del entorno de los clientes.
    \end{highlights}
\end{onecolentry}


    
    \section{Certificaciones y credenciales}

        
        \begin{twocolentry}{
            \href{https://www.credly.com/badges/7b0b48ff-3c24-4583-adb0-31f8596de99b/linked_in_profile}{Vínculo a credencial}
        }
            \textbf{Copilot for Microsoft 365 Technical Champion}\end{twocolentry}

        \vspace{0.10 cm}
        \vspace{0.2 cm}

        \begin{twocolentry}{
            \href{https://certification.kaspersky.com/api/PageOfCertificate/ca679a69-c08d-4da2-af4d-8e8486097433}{Vínculo a credencial}
        }
            \textbf{Certified Professional: Kaspersky Endpoint Security Cloud (040.16)}\end{twocolentry}

        \vspace{0.10 cm}
        \vspace{0.2 cm}

        \begin{twocolentry}{
           \href{https://certification.kaspersky.com/api/PageOfCertificate/ceec7cef-af6d-47ae-b0b6-7d8cb8dea1e7}{Vínculo a credencial}
        }
            \textbf{Certified Professional: Kaspersky Security Center for Managed Services Providers (012.12)}\end{twocolentry}

        \vspace{0.10 cm}
        \vspace{0.2 cm}
        
        \begin{twocolentry}{
           \href{https://learn.microsoft.com/api/credentials/share/es-mx/WilliamsAlejandroMedinaMrquez-4454/BE7908B9DE57559E?sharingId=1928DA4A1627BB36}{Vínculo a credencial}
        }
            \textbf{Microsoft Certified: Azure Fundamentals}\end{twocolentry}

    
\section{Tecnologías}

\begin{onecolentry}
    \textbf{Sistemas Operativos:} Windows Client \& Server, Linux (Debian-based)
\end{onecolentry}

\vspace{0.2 cm}

\begin{onecolentry}
    \textbf{Cloud \& Virtualización:} Microsoft Azure, Microsoft 365, VMware ESXi \& vSphere
\end{onecolentry}

\vspace{0.2 cm}

\begin{onecolentry}
    \textbf{Seguridad y Antimalware:} Fortigate, Palo Alto Cortex XDR, Kaspersky Security Center
\end{onecolentry}

\vspace{0.2 cm}

\begin{onecolentry}
    \textbf{Infraestructura y Redes:} Windows Server (AD DS, AD CS, File Server Services, Exchange, IIS), Dell EMC SmartFabric OS
\end{onecolentry}

\vspace{0.2 cm}

\begin{onecolentry}
    \textbf{Automatización y Scripting:} Terraform, JSON, YAML, PowerShell, Bash
\end{onecolentry}

\vspace{0.2 cm}

\begin{onecolentry}
    \textbf{Monitoreo y Herramientas:} PRTG, Jira, ManageEngine: ADManager, ADAudit, Service Desk Plus
\end{onecolentry}
 

\end{document}
